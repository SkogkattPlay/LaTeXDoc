
%%%%%%%%%%%%%%%%%%%%%%%%%%%%%%%%%%%%%%%%%%%%%%%%%%%%%%%%%%%%%%%%%%%%%%%%%%%%%%%
% SECTION 面向 Skokatt %%%%%%%%%%%%%%%%%%%%%%%%%%%%%%%%%%%%%%%%%%%%%%%%%%%%%%%%
%%%%%%%%%%%%%%%%%%%%%%%%%%%%%%%%%%%%%%%%%%%%%%%%%%%%%%%%%%%%%%%%%%%%%%%%%%%%%%%
\section{面向 Skokatt}
这是一个实现 Skokatt 的企画书,我们把 Skokatt 定义为一个高扩展的内容平台,提供我们
称之为 node 的最基本的抽象。

和其他的专门用来做 OJ 的平台不同,我们希望能够提供一个有着丰富语义的抽象层,它可
以保证我们能够得到的不仅仅是一个 OJ 系统,而更是一个内容系统:

\begin{itemize}
    \item 分布式系统,没有客户端和服务端之分,但是我们选择一个特殊的分布式系统节
        点作为其他节点的上游。更多关于分布式的内容可以查阅 \ref{fenbushi} 章。
    \item 记录所有历史,我们在最基本的抽象 node 中提供了以 commit object 为基础的
        轻量级历史系统。在基本概念上和 git 中最基本的概念保持一致。
    \item 高扩展性,可以保证后面围绕着 Skokatt 上不同类型的 node 的快速开发,正因
        为一开始设计时便确保了其构架的松耦合,才能确保越来越复杂的横向功能的基础
        上依然能够确保开发的井井有条。
    \item 向后兼容性,可以确保后期 Skokatt 即使发生了比较大的改变依然能够向后兼容
        ,这正是因为所有的 node 都保留了 Skokatt 的版本信息,可以防止因为版本更改
        而产生的错误,也可以支持后期的大幅度更改。
\end{itemize}

我们需要支持个人页面,题面,题集,和提交页面等等。而这些基本措施正是基于 node 上
进行开发的。

我们基本使用 JavaScript 来进行开发,需要先设计出能运行在桌面的 Electron 应用,再
争取使得一些 node 可以在脱离操作系统 API 的情况下运行,从而得到能够运行在浏览器
上的网页。

在后端我们需要使用 Java 进行开发,但不是最重要的地方,我们在基于分布式身份验证的
基础上进行存储身份信息,在保证能够确认身份的情况下储存一系列基本用户信息而已即可
。

基本的时间表应该如下:

\begin{center}
\begin{tabular}{cc}
    \hline
    2021/1/22 - 2021/1/27 & 学习 Java 和 Vue.js 基本语法 \\
    2021/1/28 - 2021/2/10 & 相关的 Electron 开发 \\
    2021/2/10 - 2021/2/20 & 相关的 Java 后端开发 \\
    2021/2/20 - 2021/2/30 & 在浏览器中运行子集 \\
    \hline
\end{tabular}
\end{center}

