
%%%%%%%%%%%%%%%%%%%%%%%%%%%%%%%%%%%%%%%%%%%%%%%%%%%%%%%%%%%%%%%%%%%%%%%%%%%%%%%
% SECTION node 格式 %%%%%%%%%%%%%%%%%%%%%%%%%%%%%%%%%%%%%%%%%%%%%%%%%%%%%%%%%%%
%%%%%%%%%%%%%%%%%%%%%%%%%%%%%%%%%%%%%%%%%%%%%%%%%%%%%%%%%%%%%%%%%%%%%%%%%%%%%%%
\section{node 对象模型}
我们在 \ref{fbs::node} 节中提到了 node 如何储存(即 node 的所有入口储存在 %
\verb|nodes| 文件夹中,而所有的 node 的子组成部分,即各种对象则储存在 %
\verb|objects| 中。我们必须要解释 node 中三大对象,即提交对象(commit object),
树对象(tree object)和块对象(blob object)中的具体作用。

\begin{figure}[h]
\begin{center}
\begin{tikzpicture}
    % commits, about master branch
    %%%%%%%%%%%%%%%%%%%%%%%%%%%%%%%%%%%%%%%%%%%%%%%%%%%%%%%%%%%%%%%%%%%%%%%%%%%
    \begin{scope}
        \node[commit] (master@0) {};
        \node[commit] (master@1) [left of=master@0] {}
            edge [pre] (master@0);
        \node[commit] (master@2) [left of=master@1] {}
            edge [pre] (master@1);
        \node[commit] (master@3) [left of=master@2] {}
            edge [pre] (master@2);
        \node[commit] (master@4) [left of=master@3] {}
            edge [pre] (master@3);
        \node[commit] (master@5) [left of=master@4] {}
            edge [pre] (master@4);
        % commits, about sub1 branch
        \node[commit] (sub1@0) [above of=master@2] {}
            edge [pre, bend left] (master@1)
            edge [post, bend right] (master@4);
        % commits, about sub2 branch
        \node[commit] (sub2@0) [above of=master@0] {}
            edge [post, bend right] (master@1);
    \end{scope}

    % tree of commit@0
    %%%%%%%%%%%%%%%%%%%%%%%%%%%%%%%%%%%%%%%%%%%%%%%%%%%%%%%%%%%%%%%%%%%%%%%%%%%
    \begin{scope}
        \node[tree] (obj@tree1) [below of=master@0] {}
            edge [pre] (master@0);
        \node[blob] (obj@blob1) [below left of=obj@tree1] {}
            edge [pre] (obj@tree1);
        \node[tree] (obj@tree2) [below right of=obj@tree1] {}
            edge [pre] (obj@tree1);
        \node[blob] (obj@blob2) [below left of=obj@tree2] {}
            edge [pre] (obj@tree2);
        \node[blob] (obj@blob3) [below right of=obj@tree2] {}
            edge [pre] (obj@tree2);
    \end{scope}

    % UUID & node_file & etc.
    %%%%%%%%%%%%%%%%%%%%%%%%%%%%%%%%%%%%%%%%%%%%%%%%%%%%%%%%%%%%%%%%%%%%%%%%%%%
    \begin{scope}
        \node[file] (node_file) [right of=sub2@0] {node}
            edge [post, bend left] (master@0)
            edge [post] (sub2@0);
        \node[UUID] (UUID) [above of=node_file] {UUID}
            edge [post] (node_file);
        \node[file] (HTML_file) at ([xshift=25mm] node_file.east) {HTML}
            edge [pre, dashed] (node_file);
        \node[render] (render) [below of=HTML_file] {render}
            edge [pre, bend left, dashed] (node_file)
            edge [pre, bend right] (obj@tree1)
            edge [post] (HTML_file);
    \end{scope}

    % helper
    %%%%%%%%%%%%%%%%%%%%%%%%%%%%%%%%%%%%%%%%%%%%%%%%%%%%%%%%%%%%%%%%%%%%%%%%%%%
    \begin{scope}
        \node[helper] (helper) [yshift=-5mm, below of=render, text width=3cm] {
            \tikz\node[commit]{}; commit object.\par
            \tikz\node[tree]{}; tree object.\par
            \tikz\node[blob]{}; blob object.
        };
    \end{scope}

    % background
    %%%%%%%%%%%%%%%%%%%%%%%%%%%%%%%%%%%%%%%%%%%%%%%%%%%%%%%%%%%%%%%%%%%%%%%%%%%
    \begin{pgfonlayer}{background}
        \filldraw [back_rect]
        (sub1@0.north -| master@5.west) rectangle (master@0.south -| master@0.east)
        (obj@tree1.north -| obj@blob1.west) rectangle (obj@blob3.south -| obj@blob3.east);
    \end{pgfonlayer}
\end{tikzpicture}
\end{center}
\caption{node 对象模型}
\label{node_obj_mod}
\end{figure}

而具体的实现方式可以参考样例图 \ref{node_obj_mod}。

正如 \ref{node_obj_mod} 图所指出的一样,我们通过 \verb|UUID| 可以查找到 node %
文件,它指定了它的文件版本,渲染器(render,比如说 \verb|SudaOJ:MD:1|),和以提
交对象索引所表示的历史。

提交历史使用提交对象构成的一个有向无环图来进行表示,使用内容来进行哈希,可以有效
保证了 node 历史,避免无法溯源的情况,也可以更加灵活地完成 node 的快速开发%
\footnote{这种想法完全借鉴于 git 如何管理项目历史,并经过了多个大型开源项目的开%
发的检验,比如著名的、以复杂度和高可靠性和高效性所闻名的 Linux 系统,我相信这种
基于有向无环图的历史抽象模型能够同样作用于本项目。}。

% SUBSECTION 示例 %%%%%%%%%%%%%%%%%%%%%%%%%%%%%%%%%%%%%%%%%%%%%%%%%%%%%%%%%%%%%
\subsection{示例}
举一个例子,爱丽丝想要新建一个 node,用来编辑她的关于某道 SudaOJ 上的博弈论题目
,她使用全局唯一的 \verb|UUID| 生成了一个全新的 node,没有任何历史提交,除了一个
空文件夹。她遵守着 \verb|SudaOJ:MD:1| 的渲染规则,在空文件夹下新建了一个文件 %
\verb|index.md|,并进行了相关的编辑:

\begin{lstlisting}
关于 SudaOJ:1024 题的题解
===============================================================================

我认为有手就行。
\end{lstlisting}

她将这次修改分为两次提交(对应地生成了两个提交对象,值得注意的是第一个提交对象没
有父节点)并上传到了 SudaOJ 的平台上,现在有版本历史有:
\begin{center}
    \begin{tikzpicture}
        \node[commit] (master@0) {};
        \node[commit] (master@1) [left of=master@0] {}
            edge [pre] (master@0);
    \end{tikzpicture}
\end{center}

鲍勃也注意到了爱丽丝的提交,他为她添加上了关于这道题的链接:

\begin{lstlisting}
关于 SudaOJ:1024 题的题解
===============================================================================

[原链接](oj.suda.edu.cn/node?UUID=xxx)

我认为有手就行。
\end{lstlisting}

鲍勃进行了提交,现在的版本历史为:
\begin{center}
    \begin{tikzpicture}
        \node[commit] (master@0) {};
        \node[commit] (master@1) [left of=master@0] {}
            edge [pre] (master@0);
        \node[commit] (master@2) [left of=master@1] {}
            edge [pre] (master@1);
    \end{tikzpicture}
\end{center}

与此同时,爱丽丝觉得不太礼貌,修改了文档:

\begin{lstlisting}
关于 SudaOJ:1024 题的题解
===============================================================================

我认为有手就行(不是瞧不起没有手的残疾人的意思)。
\end{lstlisting}

目前的版本历史为:
\begin{center}
    \begin{tikzpicture}
        \node[commit] (master@0) {};
        \node[commit] (master@1) [left of=master@0] {}
            edge [pre] (master@0);
        \node[commit] (master@2) [left of=master@1] {}
            edge [pre] (master@1);
        \node[commit] (sub2@0) [above of=master@0] {}
            edge [post, bend right] (master@1);
    \end{tikzpicture}
\end{center}

爱丽丝提交上去了之后发现她的 node 提示了多分支警告,爱丽丝知道 node 不应该有多
个分支,所以她更倾向于将它们进行合并:

\begin{lstlisting}
关于 SudaOJ:1024 题的题解
===============================================================================

[原链接](oj.suda.edu.cn/node?UUID=xxx)

我认为有手就行(不是瞧不起没有手的残疾人的意思)。
\end{lstlisting}

这个提交和之前的提交都不一样,它同时有两个(甚至有时候会有更多个!)父节点。我们
可以根据这个来进行作图表示当前的版本历史:

\begin{center}
    \begin{tikzpicture}
        \node[commit] (master@0) {};
        \node[commit] (master@1) [left of=master@0] {}
            edge [pre] (master@0);
        \node[commit] (master@2) [left of=master@1] {}
            edge [pre] (master@1);
        \node[commit] (sub2@0) [above of=master@0] {}
            edge [post, bend right] (master@1);
        \node[commit] (all_in_one) [right of=master@0] {}
            edge [post] (master@0)
            edge [post, bend right] (sub2@0);
    \end{tikzpicture}
\end{center}


