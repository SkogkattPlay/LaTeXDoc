
%%%%%%%%%%%%%%%%%%%%%%%%%%%%%%%%%%%%%%%%%%%%%%%%%%%%%%%%%%%%%%%%%%%%%%%%%%%%%%%
% SECTION SudaOJ 子集 %%%%%%%%%%%%%%%%%%%%%%%%%%%%%%%%%%%%%%%%%%%%%%%%%%%%%%%%%
%%%%%%%%%%%%%%%%%%%%%%%%%%%%%%%%%%%%%%%%%%%%%%%%%%%%%%%%%%%%%%%%%%%%%%%%%%%%%%%
\section{SudaOJ 子集}
基于前端都是通过 Skogkatt 来实现的,SudaOJ 子集应该实现的是一个可访问的服务。

% SudaOJ Judger %%%%%%%%%%%%%%%%%%%%%%%%%%%%%%%%%%%%%%%%%%%%%%%%%%%%%%%%%%%%%%%
\subsection{SudaOJ Judger}

SudaOJ Judger 是 SudaOJ 服务群中用来判题的核心方法,它接收一系列参数,并在参数下
在 docker 指定的 Linux 环境下运行,并返回相关的结果对象。

\subsubsection{对外接口}
我们实现一个用来判题的 Judger\footnote{封装一个开源 C 语言版本的 Judger - %
https://github.com/QingdaoU/Judger},它需要接受如表 \ref{sudaoj::judger::args} %
所示的参数,并返回运行结果对象:

\begin{table}[h]
\begin{center}
\begin{tabular}{lp{6cm}}
    \toprule
    max\_cpu\_time      & CPU 最大运行时间                                  \\
    max\_real\_time     & 最大实际运行时间(受制于 CPU 在运行时会切换进程%
                          ,故实际运行时间不能等同于程序运行时间)          \\
    max\_memory         & 最大使用内存                                      \\
    memory\_limit\_check\_only & 只检查内存而不给内存设限                   \\
    max\_stack          & 最深栈数                                          \\
    max\_process\_number &  最大进程数                                      \\
    output\_size        & 输出限制                                          \\
    \midrule
    exe\_path           & 可执行二进制文件位置                              \\
    input\_path         & 标准输入重定向文件位置                            \\
    output\_path        & 标准输出重定向文件位置                            \\
    error\_path         & 标准错误重定向文件位置                            \\
    \midrule
    args                & 接收命令行参数                                    \\
    env                 & 环境                                              \\
    \midrule
    log\_path           & 日记文件位置                                      \\
    seccomp\_rule\_name & seccomp 安全规则(沙盒化)                        \\
    \midrule
    uid                 & 程序执行 uid                                      \\
    gid                 & 程序执行 gid                                      \\
    \bottomrule
\end{tabular}
\end{center}
\caption{SudaOJ Judger 接收参数}
\label{sudaoj::judger::args}
\end{table}

需要注意的是,测量进程运行时间离不开操作系统的支持,而我们需要用 C/C++ 编写基于
Linux 的接口,并在外面利用 JNI(Java Native Interface)包装为一个 Judger 类。

Judger 类需要接受参数,调用 JNI 并返回相关的数据。

\subsubsection{底层运行原理}

本身它是封装了开源 C 语言版本的 Judger,调用 \verb|runner.c| 中的 \verb|run| 函
数,即:
\begin{lstlisting}
run(struct config *_config, struct result *_result)
\end{lstlisting}
从而进行运行程序、监控和返回运行结果。

在运行过程中调用 Linux 系统调用 \verb|fork| 分为两个互不干扰的进程,其中一个调用 %
\verb|child_process| 以在 \verb|_config| 的限制下运行,另外一个则对运行 %
\verb|child_process| 的进程进行监视,包含在指定时间内中断进程,对子进程的运行结
果来为 \verb|*_result| 变量赋值,从而实现运行。

% SudaOJ 用户 %%%%%%%%%%%%%%%%%%%%%%%%%%%%%%%%%%%%%%%%%%%%%%%%%%%%%%%%%%%%%%%%%
\subsection{SudaOJ 用户}
包含用户认证等用户系统基本组件。

对于分布式系统而言,大部分服务是可以离线进行的,但是如果需要获取 SudaOJ 的服务的
话,就必须要给出指定的令牌,以在特定的权限范围能获得访问、更改 node 的权力。比如
说如果希望访问 SudaOJ 来运行程序的话,就必须登录 SudaOJ 以要有 SudaOJ 普通权限令
牌,否则 SudaOJ 有权拒绝运行。

在 \ref{fbs::jiekou} 章中我们知道,Skogkatt 系统中为第三方服务留出了一个接口,我
们可以在和 \verb|SudaOJ| 上游服务器中认证后保留该令牌。

我们可以使用 JWT 来完成认证系统。综合来说我们有以下步骤:
\begin{enumerate}
    \item 得到登录 SudaOJ 的 node,在上面输入用户名和密码,然后把用户名和哈希之后
        的密码传入给 SudaOJ 的后端,后端返回一个 JWT 令牌,把令牌保存在 Skogkatt %
        中的 \verb|Skogkatt.ENV.SudaOJ.JWT| 中。
    \item 访问另外一个需要高权限才能请求的 node,而本地没有这个对象的时候,将 JWT %
        附带着去请求 SudaOJ 获取,SudaOJ 根据 JWT 令牌从而进行下载。
\end{enumerate}

值得注意的是,Skogkatt 虽然是分布式系统,但是请求 node 文件、object 文件等却不是由 %
Skogkatt 自己搜索的,而必须要依赖于手动更新和维护上游位置,换言之这是一种弱联网的
系统,即所承担的工作更多在本地的开发和更新上游 Skogkatt 库(当然上游 Skogkatt 库
不只是 SudaOJ 所维护的 Skogkatt 库,还可以从别人的 Skogkatt 库拉取,如果双方都在
网络中始终保持联系的话)。

