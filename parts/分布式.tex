
%%%%%%%%%%%%%%%%%%%%%%%%%%%%%%%%%%%%%%%%%%%%%%%%%%%%%%%%%%%%%%%%%%%%%%%%%%%%%%%
% SECTION 分布式 %%%%%%%%%%%%%%%%%%%%%%%%%%%%%%%%%%%%%%%%%%%%%%%%%%%%%%%%%%%%%%
%%%%%%%%%%%%%%%%%%%%%%%%%%%%%%%%%%%%%%%%%%%%%%%%%%%%%%%%%%%%%%%%%%%%%%%%%%%%%%%
\section{分布式} \label{fenbushi}

为了支持一个高扩展的 OJ 系统,我们需要支持分布式系统,提高所有节点的参与度,总体
来说采用基于分布式的原因如下:

\begin{itemize}
    \item 所有用户可以(甚至是离线的情况下!)对内容进行修改(但是上游是否采用需
        要通过一系列决策:包括拉取申请的提交者的 IP 是否在封闭期间?提交的新内
        容的文件格式是否指定需要审查?指定文件是否默认不可更改?)
\end{itemize}

% SUBSECTION 分布式系统 %%%%%%%%%%%%%%%%%%%%%%%%%%%%%%%%%%%%%%%%%%%%%%%%%%%%%%%
\subsection{分布式系统}
SudaOJ 应该建立在一个分布式系统上。为了实现分布式存储的需求,有两种不同的存储方
案:

\begin{itemize}
    \item 类似 Git 的分布式存储,基于操作系统的文件系统,使用相对路径完成定位。
        同时所有的节点地位相同,但是可以选取一个作为上游进行集中式存储。
    \item 使用类似 Ceph 的分布式数据库储存系统,可以支持大规模扩展和较高性能。
\end{itemize}

综合考虑我们选择第一个,这是因为以下的这两点原因:
\begin{itemize}
    \item 易于实现和维护,开发者和维护者只需要了解 SudaOJ 的基本概念和编程语言
        所需要的访问操作系统的文件系统 API 的相关知识即可。
    \item 更高的兼容性。为了应对每个用户可能不一样的操作系统,使用文件系统来进
        行存储可以避免数据库软件在不同的操作系统上分布式数据库的兼容性不同而带
        来的一系列问题。
\end{itemize}

% SUBSECTION 全局唯一编码 %%%%%%%%%%%%%%%%%%%%%%%%%%%%%%%%%%%%%%%%%%%%%%%%%%%%%
\subsection{全局唯一编码}
我们将所有的数据都抽象为一个 node,提出在 SudaOJ 中“一切皆 node”,在此基础上我们
需要保证所有的 node 有着全局唯一的编码,以避免冲突。我们有两种不同的 node 编码方
式:
\begin{itemize}
    \item 类似 Git 的哈希值,由内容决定编码。
    \item 使用标准 UUID,可以保证所有信息的编码两两互不相同。
\end{itemize}

我们采取第二种编码,这是因为和 Git 不一样,我们不可能在本地储存下所有的信息,其
他的 node 我们需要通过 UUID 向上游请求拉取,而因此 node 之间的互相联系的需求,就
决定了即使内容发生变化,编码也不能发生改动。

但是我们的 UUID 只是用来进行标记 node 的,node 中的具体内容并不需要一定要容纳在
文件中,我们把 node 对应的文件内容存储在 \verb|objects| 对象库里,对象库里大体上
分为三种对象:分别为提交对象(commit object),树对象(tree object)和块对象(
blob object)。

而对象的哈希我们则采用第一种方法,这种好处完全不言自喻。

% SUBSECTION node 格式 %%%%%%%%%%%%%%%%%%%%%%%%%%%%%%%%%%%%%%%%%%%%%%%%%%%%%%%%
\subsection{node 格式} \label{fbs::node}
考虑到向后兼容等一系列要求,我们让 node 的格式作为一个文件夹暴露出来,举一个例
子:

\begin{lstlisting}
.SudaOJ
    +- nodes
    |   +- xx
    |       +- xxxxxx-xxxx-Mxxx-Nxxx-xxxxxxxxxxxx
    +- objects
    |   +- xx1
    |   |   +- xxxxxxxxxxxxxxx1
    |   +- xx2
    |       +- xxxxxxxxxxxxxxx2
    +- ...
\end{lstlisting}

在数据根文件夹 \verb|.SudaOJ| 下储存着 node 库 \verb|nodes|,每一个 node 都由一
个全局唯一的 UUID 码标识,比如 UUID 码 %
\verb|xxxxxxxx-xxxx-Mxxx-Nxxx-xxxxxxxxxxxx| 标识着其对应的对象,可以通过对应的
路径进行访问。而每一个对象对应的文件内容应该如下:

\begin{lstlisting}
1
SudaOJ:MD:1

xx1xxxxxxxxxxxxxxx1
\end{lstlisting}

其中第一行为版本号,之后分别是解析格式,上游和入口提交,每一行都使用 ASCII 码中
的回车符 \verb|\n| 进行分割(值得注意的是在示例中并没有上游服务,所以说该处是一
个空行)。

其中第一行为十六进制数字,标识着系统版本号,截止本文时默认为 \verb|1|;解析格式
应该约定俗成地使用符号 `\verb|:|' 将各个命名空间隔开,最后一部分约定为格式版本,
比如说所有 SudaOJ 自定义的格式都是位于命名空间 \verb|SudaOJ| 中的,当前的文件使
用 \verb|SudaOJ:MD:1| 规则来进行解析生成对应 HTML 文件。

node 应该支持以下格式:
\begin{itemize}
    \item \verb|SudaOJ:MD:1|。
    \item \verb|SudaOJ:Question:1|。
\end{itemize}

node 文件中除了前两行,其余的代表了所有的提交对象(commit object,储存在 %
\verb|objects|),

% SUBSECTION 接口 %%%%%%%%%%%%%%%%%%%%%%%%%%%%%%%%%%%%%%%%%%%%%%%%%%%%%%%%%%%%%
\subsection{ENV 接口} \label{fbs::jiekou}
为了方便扩展,我们在运行式也会维护一个名为 \verb|Skogkatt.ENV| 的变量,它是一个
哈希表,可以通过这个为不同的服务留下全局接口。

比如说我们可以在 \verb|Skogkatt.ENV.SudaOJ| 中可以存储下互相认证之后留下的短期令
牌,从而可以因此访问和修改上游,即 \verb|SudaOJ| 服务提供的在权限范围内的接口。

